\documentclass[SE,authoryear,toc]{lsstdoc}
% lsstdoc documentation: https://lsst-texmf.lsst.io/lsstdoc.html
\input{meta}

% Package imports go here.

% Local commands go here.

%If you want glossaries
%\input{aglossary.tex}
%\makeglossaries

\title{SIT-Com Observing Workflow Definition Working Group Charge}

% Optional subtitle
% \setDocSubtitle{A subtitle}

\author{%
Sandrine Thomas (She/Her)
}

\setDocRef{SITCOMTN-031}
\setDocUpstreamLocation{\url{https://github.com/lsst-sitcom/sitcomtn-031}}

\date{\vcsDate}

% Optional: name of the document's curator
% \setDocCurator{The Curator of this Document}

\setDocAbstract{%
The charge of the SIT-Com workflow working group is developed in response to the following JIRA ticket: https://jira.lsstcorp.org/browse/SITCOM-172

We need a system to link observing tasks, scripts/notebooks, run planning, analysis, and ultimately reports. This could be somewhat analogous to the LVV project in jira, combined with some other tools (e.g. technotes). 

A proposed system and workflow can be found at https://sitcomtn-019.lsst.io/

}

% Change history defined here.
% Order: oldest first.
% Fields: VERSION, DATE, DESCRIPTION, OWNER NAME.
% See LPM-51 for version number policy.
\setDocChangeRecord{%
  \addtohist{1}{YYYY-MM-DD}{Unreleased.}{Sandrine Thomas (She/Her)}
}


\begin{document}

% Create the title page.
\maketitle
% Frequently for a technote we do not want a title page  uncomment this to remove the title page and changelog.
% use \mkshorttitle to remove the extra pages

% ADD CONTENT HERE
% You can also use the \input command to include several content files.

\appendix
% Include all the relevant bib files.
% https://lsst-texmf.lsst.io/lsstdoc.html#bibliographies
\section{References} \label{sec:bib}
\renewcommand{\refname}{} % Suppress default Bibliography section
\bibliography{local,lsst,lsst-dm,refs_ads,refs,books}

% Make sure lsst-texmf/bin/generateAcronyms.py is in your path
\section{Acronyms} \label{sec:acronyms}
\input{acronyms.tex}
% If you want glossary uncomment below -- comment out the two lines above
%\printglossaries





\end{document}
