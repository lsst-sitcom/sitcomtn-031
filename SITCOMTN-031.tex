\documentclass[SE,authoryear,toc]{lsstdoc}

% lsstdoc documentation: https://lsst-texmf.lsst.io/lsstdoc.html

\input{meta}

% Package imports go here.

% Local commands go here.

%If you want glossaries
%\input{aglossary.tex}
%\makeglossaries

\title{SIT-Com Observing Workflow Definition Working Group Charge}

% Optional subtitle
 \setDocSubtitle{A subtitle}

\author{%
Sandrine Thomas (She/Her)
}

\setDocRef{SITCOMTN-031}
\setDocUpstreamLocation{\url{https://github.com/lsst-sitcom/sitcomtn-031}}

\date{\vcsDate}

% Optional: name of the document's curator
% \setDocCurator{The Curator of this Document}

\setDocAbstract{%
The charge of the SIT-Com workflow working group is developed in response to the following JIRA ticket: https://jira.lsstcorp.org/browse/SITCOM-172
We need a system to link observing tasks, scripts/notebooks, run planning, analysis, and ultimately reports. This could be somewhat analogous to the LVV project in jira, combined with some other tools (e.g. technotes). 
A proposed system and workflow for night observing can be found at https://sitcomtn-019.lsst.io/
}

% Change history defined here.
% Order: oldest first.
% Fields: VERSION, DATE, DESCRIPTION, OWNER NAME.
% See LPM-51 for version number policy.
%\setDocChangeRecord{%
%  \addtohist{1}{YYYY-MM-DD}{Unreleased.}{Sandrine Thomas (She/Her)}
%}


\begin{document}

% Create the title page.
\maketitle
% Frequently for a technote we do not want a title page  uncomment this to remove the title page and changelog.
% use \mkshorttitle to remove the extra pages

% ADD CONTENT HERE
% You can also use the \input command to include several content files.


\section{Introduction}
The purpose of this document is to set-up a working group along with its charge to define the observatory workflows appropriate for commissioning with an eye on operation. The working group will start from the 24th cycle put together for the Observatory Operation Department in operation and extend it to the entire observatory. 
For the night observing sequence, the group will use the observing task management workflow proposal, https://sitcomtn-019.lsst.io/. 
Another existing element that needs a more refined workflow and prioritization is the operator issue reporting.

\section{Working Group Scope}
The scope of this work is to define the who, when and what. This means what information is required by whom to conduct the required daily or nightly activities. How is this information communicated and how do we ensure that the tasks have been completed before the next cycle. 
There are different length cycles that need to be taken into account: 
- The instant cycle 
- The daily cycle
- The longer term cycle (weekly?) linked to preventive maintenance
Jira exists as a tool to allow the transfer of information. It is in scope of this working group to define a more efficient workflow and prioritization process. 

The working group will deliver their report for May 31st 2022.

\section{Charge}
The working group will answer the following questions:
\begin{enumerate}
\item{Summarize the status of the workflow: 24h definition, Jira OBS project, operation rehearsals}
\item{What information is needed to be conveyed from one activity to another? This includes the definition of the observing plans, the down stream processing, required fixes...} 
\item{What are the roles and responsibilities of the people in charge of transferring information and making decision at a given time?}
\item{What tooling (existing or new) are needed  to enable this workflow?}
\item{How do we handle the notebooks created for commissioning / testing? What is the process to archive them?}
\end{enumerate}

In the execution of this charge the Working Group is requested to reach out to other stakeholders in their deliberations and evaluations.
The outcome from this working will be a document that reports on the charge items indicated above.

\section{Working Group Members}
\begin{enumerate}
\item Patrick Ingraham (Chair)
\item Robert
\item Frossie
\item Alysha
\item Holger
\end{enumerate}

%Colin Slater? or Jeff Carlin?
%Chuck
%Michael Reuter

\appendix
% Include all the relevant bib files.
% https://lsst-texmf.lsst.io/lsstdoc.html#bibliographies
\section{References} \label{sec:bib}
\renewcommand{\refname}{} % Suppress default Bibliography section
\bibliography{local,lsst,lsst-dm,refs_ads,refs,books}

% Make sure lsst-texmf/bin/generateAcronyms.py is in your path
\section{Acronyms} \label{sec:acronyms}
\input{acronyms.tex}
% If you want glossary uncomment below -- comment out the two lines above
%\printglossaries

\end{document}
